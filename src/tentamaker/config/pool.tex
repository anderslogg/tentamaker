% A.1.1
Bestäm dubbelintegralen av funktionen $f(x, y) = 64(x^2 + y^2)/\pi$ över området
som begränsas av tredje kvadranten av enhets\-cirkeln och $r > \tfrac12$.
% $\frac{15}{2} = \num{7.5}$
Polära koordinater ger
\begin{align*}
  \int_{\pi}^{3\pi/2}\int_{1/2}^1 \frac{64r^2}{\pi} r\dr\dphi
  = \frac{64}{\pi}\frac{\pi}{2} \int_{1/2}^1 r^3\dr
  = 32 \cdot \frac{1}{4} \cdot \left(1 - \frac1{16}\right) = \frac{15}2
\end{align*}
%

% A.2.1
Bestäm dubbelintegralen av funktionen $f(x, y) = 64(x^2 + y^2)/\pi$ över området
som begränsas av tredje kvadranten av enhets\-cirkeln och $r > \tfrac12$.
% $\frac{15}{2} = \num{7.5}$
Polära koordinater ger
\begin{align*}
  \int_{\pi}^{3\pi/2}\int_{1/2}^1 \frac{64r^2}{\pi} r\dr\dphi
  = \frac{64}{\pi}\frac{\pi}{2} \int_{1/2}^1 r^3\dr
  = 32 \cdot \frac{1}{4} \cdot \left(1 - \frac1{16}\right) = \frac{15}2
\end{align*}
%

% A.3.1
Bestäm dubbelintegralen av funktionen $f(x, y) = 64(x^2 + y^2)/\pi$ över området
som begränsas av tredje kvadranten av enhets\-cirkeln och $r > \tfrac12$.
% $\frac{15}{2} = \num{7.5}$
Polära koordinater ger
\begin{align*}
  \int_{\pi}^{3\pi/2}\int_{1/2}^1 \frac{64r^2}{\pi} r\dr\dphi
  = \frac{64}{\pi}\frac{\pi}{2} \int_{1/2}^1 r^3\dr
  = 32 \cdot \frac{1}{4} \cdot \left(1 - \frac1{16}\right) = \frac{15}2
\end{align*}
%

% C.1.1
Vilket tal är det tänkt att programmet skall räkna ut (variabeln \texttt{y})? \\
\begin{python}
from numpy.linalg import *

x = 1
y = 1

for k in range(1000):
    A = [[2*x, 2*y], [-2*x, 1]]
    b = [x**2 + y**2 - 42, y - x**2]
    sol = solve(A, b)
    x = x - sol[0]
    y = y - sol[1]

print(y)
\end{python}
% $6$
Programmet beräknar en lösning till ekvationssystemet
\begin{align*}
  x^2 + y^2 &= 42 \\
  y &= x^2
\end{align*}
med Newtons metod. Lösningarna är $(x, y) = (\pm\sqrt{6}, 6)$ och således $y = 6
$.

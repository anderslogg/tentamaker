% A.1.1
Bestäm dubbelintegralen av funktionen $f(x, y) = 64(x^2 + y^2)/\pi$ över området
som begränsas av tredje kvadranten av enhets\-cirkeln och $r > \tfrac12$.
% $\frac{15}{2} = \num{7.5}$
Polära koordinater ger
\begin{align*}
  \int_{\pi}^{3\pi/2}\int_{1/2}^1 \frac{64r^2}{\pi} r\dr\dphi
  = \frac{64}{\pi}\frac{\pi}{2} \int_{1/2}^1 r^3\dr
  = 32 \cdot \frac{1}{4} \cdot \left(1 - \frac1{16}\right) = \frac{15}2
\end{align*}
%

% A.2.1
Bestäm dubbelintegralen av funktionen $f(x, y) = 64(x^2 + y^2)/\pi$ över området
som begränsas av tredje kvadranten av enhets\-cirkeln och $r > \tfrac12$.
% $\frac{15}{2} = \num{7.5}$
Polära koordinater ger
\begin{align*}
  \int_{\pi}^{3\pi/2}\int_{1/2}^1 \frac{64r^2}{\pi} r\dr\dphi
  = \frac{64}{\pi}\frac{\pi}{2} \int_{1/2}^1 r^3\dr
  = 32 \cdot \frac{1}{4} \cdot \left(1 - \frac1{16}\right) = \frac{15}2
\end{align*}
%

% A.3.1
Bestäm dubbelintegralen av funktionen $f(x, y) = 64(x^2 + y^2)/\pi$ över området
som begränsas av tredje kvadranten av enhets\-cirkeln och $r > \tfrac12$.
% $\frac{15}{2} = \num{7.5}$
Polära koordinater ger
\begin{align*}
  \int_{\pi}^{3\pi/2}\int_{1/2}^1 \frac{64r^2}{\pi} r\dr\dphi
  = \frac{64}{\pi}\frac{\pi}{2} \int_{1/2}^1 r^3\dr
  = 32 \cdot \frac{1}{4} \cdot \left(1 - \frac1{16}\right) = \frac{15}2
\end{align*}
%

Tentamen består av fyra delar:
\begin{itemize}
\item
  Del A: Räkneuppgifter ($12 \times 3\mathrm{p} = 36\mathrm{p}$)
\item
  Del B: Teorifrågor ($4 \times 2\mathrm{p} = 8\mathrm{p}$)
\item
  Del C: Programmering ($2 \times 3\mathrm{p} = 6\mathrm{p}$)
\end{itemize}
Tillsammans ger dessa uppgifter maximalt 50p.
Till detta läggs de bonuspoäng som tjänats ihop under kursens gång.
Betygsgränser är 20p (betyg 3), 30p (betyg 4) och 40p (betyg 5)
för det sammanlagda resultatet. Poäng ges endast för rätt svar;
delpoäng ges endast i undantagsfall.

\bigskip

\hrule

\smallskip

\textbf{OBS!} Notera följande enkla men \textbf{VIKTIGA} regler för inmatning av svar.

\textbf{Underlåtenhet att följa dessa regler resulterar i 0p på uppgiften!}

\begin{enumerate}
\item
  Ange svaret som ett heltal eller ett decimaltal.
\item
  Avrunda korrekt till 3 decimaler om svaret har fler än 3 decimaler.
\item
  Svar på teorifrågorna (13-16) skall ges som ett eller två ord.
\end{enumerate}
